%%%%%%%%%%%%%%%%%%%%%%%%%%%%%%%%%%%%%%%%%%%%%%%%%%%%%%%%%%%%%%%%%%%%%%%%%
%%
%W  info.tex             POLENTA documentation            Bjoern Assmann
%W                                                     
%W                                                     
%W                                                       
%%
%H  @(#)$Id$
%%
%Y 2003
%%%%%%%%%%%%%%%%%%%%%%%%%%%%%%%%%%%%%%%%%%%%%%%%%%%%%%%%%%%%%%%%%%%%%%%%%
\Chapter{Information Messages}

It is possible to get informations about the status of the computation of the 
functions of Chapter 2 of this manual.

%%%%%%%%%%%%%%%%%%%%%%%%%%%%%%%%%%%%%%%%%%%%%%%%%%%%%%%%%%%%%%%%%%%%%%%%%
\Section{Info Class}

\> `InfoPolenta'{InfoPolenta}
 
is the Info class of the {\Polenta} package (for more details on the Info mechanism see Section~"ref:Info Functions" of the 
{\GAP} Reference Manual). 
With the help of the function 
`SetInfoLevel(InfoPolenta,<level>)' you can change 
the info level of `InfoPolenta'. 
\beginlist
\item{--}
  If  `InfoLevel( InfoPolenta )' is equal to 0 
 then no information 
  messages are displayed. 
\item{--}
  If `InfoLevel( InfoPolenta )' is equal to 1 then basic informations
  about the process are provided. For further background on the displayed 
  informations we refer to  \cite{Assmann} (publicly available via the 
  Internet address `http://cayley.math.nat.tu-bs.de/software/assmann/').
\item{--}
  If `InfoLevel( InfoPolenta )' is equal to 2 then, in addition to the 
  basic information, the generators of computed subgroups and module series
  are displayed. 
\endlist


%%%%%%%%%%%%%%%%%%%%%%%%%%%%%%%%%%%%%%%%%%%%%%%%%%%%%%%%%%%%%%%%%%%%%%%%%
\Section{Example}

\beginexample
gap> SetInfoLevel( InfoPolenta, 1 );

gap> PcpGroupByMatGroup( PolExamples(11) );
#I  Determine a constructive polycyclic sequence
    for the input group ...
#I
#I  Chosen admissible prime: 3
#I
#I  Determine a constructive polycyclic sequence
    for the image under the p-congruence homomorphism ...
#I  finished.
#I  Finite image has relative orders [ 3, 2, 3, 3, 3 ].
#I
#I  Compute normal subgroup generators for the kernel
    of the p-congruence homomorphism ...
#I  finished.
#I
#I  Compute the radical series ...
#I  finished.
#I  The radical series has length 4.
#I
#I  Compute the composition series ...
#I  finished.
#I  The composition series has length 5.
#I
#I  Compute a constructive polycyclic sequence
    for the induced action of the kernel to the composition series ...
#I  finished.
#I  This polycyclic sequence has relative orders [  ].
#I
#I  Calculate normal subgroup generators for the
    unipotent part ...
#I  finished.
#I
#I  Determine a constructive polycyclic  sequence
    for the unipotent part ...
#I  finished.
#I  The unipotent part has relative orders
#I  [ 0, 0, 0 ].
#I
#I  ... computation of a constructive
    polycyclic sequence for the whole group finished.
#I
#I  Compute the relations of the polycyclic
    presentation of the group ...
#I  Compute power relations ...
#I  ... finished.
#I  Compute conjugation relations ...
#I  ... finished.
#I  Update polycyclic collector ...
#I  ... finished.
#I  finished.
#I
#I  Construct the polycyclic presented group ...
#I  finished.
#I
Pcp-group with orders [ 3, 2, 3, 3, 3, 0, 0, 0 ]


gap> SetInfoLevel( InfoPolenta, 2 );

gap> PcpGroupByMatGroup( PolExamples(11) );
#I  Determine a constructive polycyclic sequence
    for the input group ...
#I
#I  Chosen admissible prime: 3
#I
#I  Determine a constructive polycyclic sequence
    for the image under the p-congruence homomorphism ...
#I  finished.
#I  Finite image has relative orders [ 3, 2, 3, 3, 3 ].
#I
#I  Compute normal subgroup generators for the kernel
    of the p-congruence homomorphism ...
#I  finished.
#I  The normal subgroup generators are
#I  [ [ [ 1, -3/2, 0, 0 ], [ 0, 1, 0, 0 ], [ 0, 0, 1, 3 ], [ 0, 0, 0, 1 ] ],
  [ [ 1, 0, 0, 24 ], [ 0, 1, 0, 0 ], [ 0, 0, 1, 0 ], [ 0, 0, 0, 1 ] ],
  [ [ 1, 3, 3, 15 ], [ 0, 1, 0, 6 ], [ 0, 0, 1, -6 ], [ 0, 0, 0, 1 ] ],
  [ [ 1, 3, 3, 9 ], [ 0, 1, 0, 6 ], [ 0, 0, 1, -6 ], [ 0, 0, 0, 1 ] ],
  [ [ 1, 3/2, 3/2, 3/2 ], [ 0, 1, 0, 3 ], [ 0, 0, 1, -3 ], [ 0, 0, 0, 1 ] ],
  [ [ 1, -3/2, 9/2, -69/2 ], [ 0, 1, 0, 9 ], [ 0, 0, 1, 3 ], [ 0, 0, 0, 1 ] ]
    , [ [ 1, 0, 0, -24 ], [ 0, 1, 0, 0 ], [ 0, 0, 1, 0 ], [ 0, 0, 0, 1 ] ],
  [ [ 1, -3, -3, -9 ], [ 0, 1, 0, -6 ], [ 0, 0, 1, 6 ], [ 0, 0, 0, 1 ] ],
  [ [ 1, -3, -3, -15 ], [ 0, 1, 0, -6 ], [ 0, 0, 1, 6 ], [ 0, 0, 0, 1 ] ],
  [ [ 1, -3, 0, 9 ], [ 0, 1, 0, 0 ], [ 0, 0, 1, 6 ], [ 0, 0, 0, 1 ] ],
  [ [ 1, -3, -3, -9 ], [ 0, 1, 0, -6 ], [ 0, 0, 1, 6 ], [ 0, 0, 0, 1 ] ],
  [ [ 1, -3, 0, 9 ], [ 0, 1, 0, 0 ], [ 0, 0, 1, 6 ], [ 0, 0, 0, 1 ] ],
  [ [ 1, -3/2, -3/2, -9/2 ], [ 0, 1, 0, -3 ], [ 0, 0, 1, 3 ], [ 0, 0, 0, 1 ]
     ],
  [ [ 1, -3, -3, -12 ], [ 0, 1, 0, -6 ], [ 0, 0, 1, 6 ], [ 0, 0, 0, 1 ] ],
  [ [ 1, 3, -3/2, -21 ], [ 0, 1, 0, -3 ], [ 0, 0, 1, -6 ], [ 0, 0, 0, 1 ] ],
  [ [ 1, 3/2, 3/2, 9/2 ], [ 0, 1, 0, 3 ], [ 0, 0, 1, -3 ], [ 0, 0, 0, 1 ] ] ]
#I
#I  Compute the radical series ...
#I  finished.
#I  The radical series has length 4.
#I  The radical series is
#I  [ [ [ 1, 0, 0, 0 ], [ 0, 1, 0, 0 ], [ 0, 0, 1, 0 ], [ 0, 0, 0, 1 ] ],
  [ [ 0, 1, 0, 0 ], [ 0, 0, 1, 0 ], [ 0, 0, 0, 1 ] ], [ [ 0, 0, 0, 1 ] ],
  [  ] ]
#I
#I  Compute the composition series ...
#I  finished.
#I  The composition series has length 5.
#I  The composition series is
#I  [ [ [ 1, 0, 0, 0 ], [ 0, 1, 0, 0 ], [ 0, 0, 1, 0 ], [ 0, 0, 0, 1 ] ],
  [ [ 0, 1, 0, 0 ], [ 0, 0, 1, 0 ], [ 0, 0, 0, 1 ] ],
  [ [ 0, 0, 1, 0 ], [ 0, 0, 0, 1 ] ], [ [ 0, 0, 0, 1 ] ], [  ] ]
#I
#I  Compute a constructive polycyclic sequence
    for the induced action of the kernel to the composition series ...
#I  finished.
#I  This polycyclic sequence has relative orders [  ].
#I
#I  Calculate normal subgroup generators for the
    unipotent part ...
#I  finished.
#I  The normal subgroup generators for the unipotent part are
#I  [ [ [ 1, -3/2, 0, 0 ], [ 0, 1, 0, 0 ], [ 0, 0, 1, 3 ], [ 0, 0, 0, 1 ] ],
  [ [ 1, 0, 0, 24 ], [ 0, 1, 0, 0 ], [ 0, 0, 1, 0 ], [ 0, 0, 0, 1 ] ],
  [ [ 1, 3, 3, 15 ], [ 0, 1, 0, 6 ], [ 0, 0, 1, -6 ], [ 0, 0, 0, 1 ] ],
  [ [ 1, 3, 3, 9 ], [ 0, 1, 0, 6 ], [ 0, 0, 1, -6 ], [ 0, 0, 0, 1 ] ],
  [ [ 1, 3/2, 3/2, 3/2 ], [ 0, 1, 0, 3 ], [ 0, 0, 1, -3 ], [ 0, 0, 0, 1 ] ],
  [ [ 1, -3/2, 9/2, -69/2 ], [ 0, 1, 0, 9 ], [ 0, 0, 1, 3 ], [ 0, 0, 0, 1 ] ]
    , [ [ 1, 0, 0, -24 ], [ 0, 1, 0, 0 ], [ 0, 0, 1, 0 ], [ 0, 0, 0, 1 ] ],
  [ [ 1, -3, -3, -9 ], [ 0, 1, 0, -6 ], [ 0, 0, 1, 6 ], [ 0, 0, 0, 1 ] ],
  [ [ 1, -3, -3, -15 ], [ 0, 1, 0, -6 ], [ 0, 0, 1, 6 ], [ 0, 0, 0, 1 ] ],
  [ [ 1, -3, 0, 9 ], [ 0, 1, 0, 0 ], [ 0, 0, 1, 6 ], [ 0, 0, 0, 1 ] ],
  [ [ 1, -3, -3, -9 ], [ 0, 1, 0, -6 ], [ 0, 0, 1, 6 ], [ 0, 0, 0, 1 ] ],
  [ [ 1, -3, 0, 9 ], [ 0, 1, 0, 0 ], [ 0, 0, 1, 6 ], [ 0, 0, 0, 1 ] ],
  [ [ 1, -3/2, -3/2, -9/2 ], [ 0, 1, 0, -3 ], [ 0, 0, 1, 3 ], [ 0, 0, 0, 1 ]
     ],
  [ [ 1, -3, -3, -12 ], [ 0, 1, 0, -6 ], [ 0, 0, 1, 6 ], [ 0, 0, 0, 1 ] ],
  [ [ 1, 3, -3/2, -21 ], [ 0, 1, 0, -3 ], [ 0, 0, 1, -6 ], [ 0, 0, 0, 1 ] ],
  [ [ 1, 3/2, 3/2, 9/2 ], [ 0, 1, 0, 3 ], [ 0, 0, 1, -3 ], [ 0, 0, 0, 1 ] ] ]
#I
#I  Determine a constructive polycyclic  sequence
    for the unipotent part ...
#I  finished.
#I  The unipotent part has relative orders
#I  [ 0, 0, 0 ].
#I
#I  ... computation of a constructive
    polycyclic sequence for the whole group finished.
#I
#I  Compute the relations of the polycyclic
    presentation of the group ...
#I  Compute power relations ...
.....
#I  ... finished.
#I  Compute conjugation relations ...
..............................................
#I  ... finished.
#I  Update polycyclic collector ...
#I  ... finished.
#I  finished.
#I
#I  Construct the polycyclic presented group ...
#I  finished.
#I
Pcp-group with orders [ 3, 2, 3, 3, 3, 0, 0, 0 ]


\endexample
%%%%%%%%%%%%%%%%%%%%%%%%%%%%%%%%%%%%%%%%%%%%%%%%%%%%%%%%%%%%%%%%%%%%%%%%%
%%
%E  Emacs . . . . . . . . . . . . . . . . . . . . . local emacs variables
%%
%%  Local Variables:
%%  fill-column:    73
%%  End:
%%












