%%%%%%%%%%%%%%%%%%%%%%%%%%%%%%%%%%%%%%%%%%%%%%%%%%%%%%%%%%%%%%%%%%%%%%%%%
%%
%W  intro.tex             POLENTA documentation            Bjoern Assmann
%W                                                     
%W                                                     
%W                                                       
%%
%H  @(#)$Id$
%%
%Y 2003
%%
%%%%%%%%%%%%%%%%%%%%%%%%%%%%%%%%%%%%%%%%%%%%%%%%%%%%%%%%%%%%%%%%%%%%%%%%%
\Chapter{Introduction}

\atindex{Polenta}{@Polenta}
\atindex{Polycyclic}{@Polycyclic}

%%%%%%%%%%%%%%%%%%%%%%%%%%%%%%%%%%%%%%%%%%%%%%%%%%%%%%%%%%%%%%%%%%
\Section{The package}

This package provides functions for the computation with matrix
groups. Let $G$ be a subgroup of $GL(d,R)$ where the ring $R$ is
either equal to $\Q,\Z$ or a finite field $\F_q$.
Then: 
 \smallskip
{\parindent=25pt
\item{$\bullet$} 
    We can test whether $G$ is solvable.
 \smallskip
\item{$\bullet$}
    If $G$ is polycyclic, then we can determine a polycyclic
    presentation for $G$. 
    \smallskip
}
\smallskip

A group $G$ which is given by a polycyclic presentation can be largely
investigated by algorithms implemented in the {\GAP}-package
Polycyclic \cite{polycyclic}. For example 
we can determine if $G$ is torsion-free
and calculate the torsion subgroup. Further we can compute the derived
series and the Hirschlength of the group $G$. Also various methods for
computations with subgroups, factorsgroups and extensions are
available.

In the case that the matrix group 
$G$ is a subgroup of $GL(d,\F_q)$ or $GL(d,\Z)$, the
group $G$ is solvable if and only if $G$ is polycyclic (see Chapter 1
in \cite{Segal}). 
Therefore, in this case, we can test if $G$ is polycyclic. 

As a by-product, the {\Polenta} package 
provides some functionality to compute certain module series for
modules of solvable groups. For example, if
$G$ is a rational polycyclic matrix group, then we can compute the 
radical series of the natural
$\Q[G]$-module $\Q^d$.  

%%%%%%%%%%%%%%%%%%%%%%%%%%%%%%%%%%%%%%%%%%%%%%%%%%%%%%%%%%%%%%%%%%%%%%%%
\Section{Polycyclic groups}

A group $G$ is called polycyclic if it has a finite subnormal
series with cyclic 
factors. It is a well-known fact that every polycyclic group is
finitely presented by a so-called polycyclic presentation (see
for example Chapter 9 in \cite{Sims} or Chapter 2 in \cite{polycyclic} ). 
In {\GAP} groups which are defined by polycyclic
 presentations are called
polycyclically presented groups, short PcpGroups.
 
The overall idea of the algorithm implemented in this package have
first been introduced 
by Ostheimer in 1996 \cite{Ostheimer}. 
In 2001 Eick presented a more detailed
version \cite{Eick}. This package contains an implementation of this
algorithm. A description of this implementation together with some
refinements and extensions can be
found in \cite{Assmann}. 

%%%%%%%%%%%%%%%%%%%%%%%%%%%%%%%%%%%%%%%%%%%%%%%%%%%%%%%%%%%%%%%%%%%%%%%%%
%%
%E






