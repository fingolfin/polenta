%%%%%%%%%%%%%%%%%%%%%%%%%%%%%%%%%%%%%%%%%%%%%%%%%%%%%%%%%%%%%%%%%%%%%%%%%
%%
%W  install.tex           POLENTA documentation            Bjoern Assmann
%W                                                     
%W                                                     
%W                                                       
%%
%H  @(#)$Id$
%%
%Y 2003
%%
%%%%%%%%%%%%%%%%%%%%%%%%%%%%%%%%%%%%%%%%%%%%%%%%%%%%%%%%%%%%%%%%%%%%%%%%%
\Chapter{Installation}

\atindex{Installation}{@Installation}
%%%%%%%%%%%%%%%%%%%%%%%%%%%%%%%%%%%%%%%%%%%%%%%%%%%%%%%%%%%%%%%%%%%%%%%%%
\Section{Installing this package}\null

The Polenta package is part of the standard distribution of {\GAP} and
so normally there should be no need to install it separately.
If by any chance it is not part of your {\GAP} distribution, then 
the standard method is to unpack	 the package into the `pkg'
directory  of your {\GAP} distribution.  This will create a `polenta'
subdirectory. 
For other non-standard options please see  Chapter~"ref:Installing a
GAP Package" of the {\GAP} Reference Manual.

Note that the GAP-Packages Alnuth and Polycyclic are needed for this package.
Normally they should be contained in your distribution. If not, 
they can be obtained at
\begintt
             http://cayley.math.nat.tu-bs.de/software/content.html
\endtt


%To create the documentation, go into the `doc' directory and type
%`make_doc'.

%%%%%%%%%%%%%%%%%%%%%%%%%%%%%%%%%%%%%%%%%%%%%%%%%%%%%%%%%%%%%%%%%%%%%%%%%%%%%
\Section{Getting and installing KASH}
                    
Note that the {\GAP} package Alnuth whose functionality is used by the Polenta
package requires the installation of KANT respectively
KASH, the shell
of the computational algebraic number theory system KANT.  KASH itself
is not part of Alnuth.  It has to be obtained and installed
independently.
                                                            
KASH is available at
\begintt
               www.math.tu-berlin.de/~kant/download.html
\endtt
Note that you have to download two files for a complete installation
of KASH.  To install version 2.4 of KASH on a Linux
system you should do the following:
\beginlist
\item{1.} Download the files
    kash_2.4.common.tar.gz and kash_2.4.1.linux.tar.gz
    into the same directory on your system.
                                                                               
\item{2.} Unpack the files using tar.  This will create a directory
    KASH_2.4 containing, amongst other files, the KASH executable called
    kash.  The place where KASH is put is independent of the place
    where the Alnuth or Polenta package is installed.
    
\endlist
                                                                               


%%%%%%%%%%%%%%%%%%%%%%%%%%%%%%%%%%%%%%%%%%%%%%%%%%%%%%%%%%%%%%%%%%%%%%%%%
\Section{Loading the Polenta package}\null

\atindex{Loading the Polenta package}{@loading the {\Polenta} package}
If the {\Polenta} Package is not already loaded 
then you have to request it explicitly. 
This  can be 
done by `LoadPackage("polenta")'.
The `LoadPackage' command is described in Section~"ref:LoadPackage"
in the {\GAP} Reference Manual.

%%%%%%%%%%%%%%%%%%%%%%%%%%%%%%%%%%%%%%%%%%%%%%%%%%%%%%%%%%%%%%%%%%%%%%%%%
\Section{Running the test suite}

    Once the package is installed, it is possible to check the correct
    installation by running the test suite of the package.

\beginexample
    gap> Read( "mygap/pkg/polenta/tst/testall.g" );
\endexample
    where `mygap' needs to be replaced with the directory where {\GAP}
    was installed. For more details on  Test Files see 
    Section~"ref:Test Files" of the 
    {\GAP} Reference Manual.

    If the test suite runs into an error, even though the packages
    Polycyclic and 
    Alnuth  and  the computational algebraic number theory system KANT
    have been correctly installed, then please send a message
    to `BjoernAssmann@gmx.net' including the error message.

%%%%%%%%%%%%%%%%%%%%%%%%%%%%%%%%%%%%%%%%%%%%%%%%%%%%%%%%%%%%%%%%%%%%%%%%%
%%
%E


