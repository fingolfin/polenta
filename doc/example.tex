%%%%%%%%%%%%%%%%%%%%%%%%%%%%%%%%%%%%%%%%%%%%%%%%%%%%%%%%%%%%%%%%%%%%%%%%%
%%
%W  example.tex           POLENTA documentation            Bjoern Assmann
%W                                                     
%W                                                     
%W                                                       
%%
%H  @(#)$Id$
%%
%Y 2003
%%
\Chapter{An example application}

In this section we outline three example computations with functions
from the previous chapter. 

%%%%%%%%%%%%%%%%%%%%%%%%%%%%%%%%%%%%%%%%%%%%%%%%%%%%%%%%%%%%%%%%%%
\Section{Presentation for rational matrix groups}

\beginexample
gap>mats := 
[ [ [ 1, 0, -1/2, 0 ], [ 0, 1, 0, 1 ], [ 0, 0, 1, 0 ], [ 0, 0, 0, 1 ] ],
  [ [ 1, 1/2, 0, 0 ], [ 0, 1, 0, 0 ], [ 0, 0, 1, 1 ], [ 0, 0, 0, 1 ] ],
  [ [ 1, 0, 0, 1 ], [ 0, 1, 0, 0 ], [ 0, 0, 1, 0 ], [ 0, 0, 0, 1 ] ],
  [ [ 1, -1/2, -3, 7/6 ], [ 0, 1, -1, 0 ], [ 0, 1, 0, 0 ], [ 0, 0, 0, 1 ] ],
  [ [ -1, 3, 3, 0 ], [ 0, 0, 1, 0 ], [ 0, 1, 0, 0 ], [ 0, 0, 0, 1 ] ] ]

gap> G := Group( mats );
<matrix group with 5 generators>

# calculate an isomorphism from G to a pcp-group
gap> nat := IsomorphismPcpGroup( G );;   
     
gap> H := Image( nat );
Pcp-group with orders [ 2, 2, 3, 5, 5, 5, 0, 0, 0 ]

gap> h := GeneratorsOfGroup( H );
[ g1, g2, g3, g4, g5, g6, g7, g8, g9]

gap> mats2 := List( h, x -> PreImage( nat, x ) );;

# take a random element of G
gap> exp :=  [ 1, 1, 1, 1, 0, 0, 0, 0, 1 ];;
gap> g := MappedVector( exp, mats2 );
[ [ -1, 17/2, -1, 173/6 ], 
  [ 0, 1, 0, -2 ], 
  [ 0, 1, -1, 2 ], 
  [ 0, 0,  0, 1 ] ]

# map g into the image of nat
gap> i := ImageElm( nat, g );
g1*g2*g3*g4

# exponent vector 
gap> Exponents( i );
[ 1, 1, 1, 1, 0, 0, 0, 0 ]

# compare the preimage with g
gap> PreImagesRepresentative( nat, i );
[ [ -1, 17/2, -1, 173/6 ], 
  [ 0, 1, 0, -2 ], 
  [ 0, 1, -1, 2 ], 
  [ 0, 0, 0, 1 ] ]

gap> last = g;
true

\endexample

%%%%%%%%%%%%%%%%%%%%%%%%%%%%%%%%%%%%%%%%%%%%%%%%%%%%%%%%%%%%%%%%%%
\Section{Modules series}

\beginexample
gap> gens := 
[ [ [ 1746/1405, 524/7025, 418/1405, -77/2810 ], 
    [ 815/843, 899/843, -1675/843, 415/281 ],
    [ -3358/4215, -3512/21075, 4631/4215, -629/1405 ],
    [ 258/1405, 792/7025, 1404/1405, 832/1405 ] ],
  [ [ -2389/2810, 3664/21075, 8942/4215, -35851/16860 ],
    [ 395/281, 2498/2529, -5105/5058, 3260/2529 ],
    [ 3539/2810, -13832/63225, -12001/12645, 87053/50580 ],
    [ 5359/1405, -3128/21075, -13984/4215, 40561/8430 ] ] ];

gap> H := Group( gens );
<matrix group with 2 generators>
     
gap> RadicalSeriesSolvableMatGroup( H );
[ [ [ 1, 0, 0, 0 ], [ 0, 1, 0, 0 ], [ 0, 0, 1, 0 ], [ 0, 0, 0, 1 ] ],
  [ [ 1, 0, 0, 79/138 ], [ 0, 1, 0, -275/828 ], [ 0, 0, 1, -197/414 ] ],
  [ [ 1, 0, -3, 2 ], [ 0, 1, 55/4, -55/8 ] ], 
  [ [ 1, 4/15, 2/3, 1/6 ] ], 
  [  ] ]
\endexample
%%%%%%%%%%%%%%%%%%%%%%%%%%%%%%%%%%%%%%%%%%%%%%%%%%%%%%%%%%%%%%%%%%
\Section{Triangularizable subgroups}

\beginexample
gap> G := PolExamples(3);
<matrix group with 2 generators>

gap> GeneratorsOfGroup( G );
[ [ [ 73/10, -35/2, 42/5, 63/2 ], 
    [ 27/20, -11/4, 9/5, 27/4 ],
    [ -3/5, 1, -4/5, -9 ], 
    [ -11/20, 7/4, -2/5, 1/4 ] ],
  [ [ -42/5, 423/10, 27/5, 479/10 ], 
    [ -23/10, 227/20, 13/10, 231/20 ],
    [ 14/5, -63/5, -4/5, -79/5 ], 
    [ -1/10, 9/20, 1/10, 37/20 ] ] ]

gap> T := TriangNormalSubgroupFiniteInd( G );
<matrix group with 2 generators>

gap> GeneratorsOfGroup( T );
[ [ [ 73/10, -35/2, 42/5, 63/2 ], 
    [ 27/20, -11/4, 9/5, 27/4 ],
    [ -3/5, 1, -4/5, -9 ], 
    [ -11/20, 7/4, -2/5, 1/4 ] ],
  [ [ -42/5, 423/10, 27/5, 479/10 ], 
    [ -23/10, 227/20, 13/10, 231/20 ],
    [ 14/5, -63/5, -4/5, -79/5 ], 
    [ -1/10, 9/20, 1/10, 37/20 ] ] ]

# so G is triangularizable!
\endexample























