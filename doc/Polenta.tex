%%%%%%%%%%%%%%%%%%%%%%%%%%%%%%%%%%%%%%%%%%%%%%%%%%%%%%%%%%%%%%%%%%%%%%%%%
%%
%W  intro.tex             POLENTA documentation            Bjoern Assmann
%W                                                     
%W                                                     
%W                                                       
%%
%H  @(#)$Id$
%%
%Y 2003
%%%%%%%%%%%%%%%%%%%%%%%%%%%%%%%%%%%%%%%%%%%%%%%%%%%%%%%%%%%%%%%%%%%%%%%%%
\Chapter{Methods for matrix groups}

%%%%%%%%%%%%%%%%%%%%%%%%%%%%%%%%%%%%%%%%%%%%%%%%%%%%%%%%%%%%%%%%%%%%%%%%%
\Section{Polycyclic presentations of matrix groups}
\label{section_present}
 
Groups defined by polycyclic presentations are called PcpGroups in
{\GAP}.
We refer to the Polycyclic manual \cite{polycyclic} for further
background.

Suppose that a collection of
matrices of $GL(d,R)$ is given, where the ring $R$ 
is either $\Q,\Z$ or a finite field.  Let $G$ be the group which is
generated by these matrices. If the group $G$ is polycyclic, then the
following functions determine a PcpGroup isomorphic to $G$.
 
\> PcpGroupByMatGroup( <G> )
\> IsomorphismPcpGroup( <G> )
 
<G> is  a subgroup of $GL(d,R)$ where $R=\Q,\Z $ or $\F_q$.
If <G> is polycyclic, then 
these functions determine a PcpGroup isomorphic to <G> and an isomorphism
onto this group. 
If <G> is not polycyclic, then there are two cases: If $R=\Z$ or $\F_q$,
then algorithms returns 'fail'. In case that $R=\Q$ the algorithm may
return 'fail' or may not terminate.

\> Image( <map> ) 
\> ImageElm( <map>, <elm> )
\> ImagesSet( <map>, <elms> )
\> PreImagesRepresentative( <map>, <pcpelm> )
 
Here <map> is an isomorphism from a polycyclic matrix group <G>
onto a PcpGroup <H> calculated
by {\tt IsomorphismPcpGroup(<G>)}.
These functions can be used to compute with such an isomorphism. 
If the input <elm>  is an element of <G>, then the function {\tt
ImageElm} can be used to compute the image of <elm> under <map>. 
If <elm> is not contained in <G>
<map>,
then the function {\tt ImageElm} returns 'fail'. 
The input <pcpelm> is an element
of <H>. 

\> IsSolvableGroup( <G> )
\> IsSolvableMatGroup( <G> )

<G> is  a subgroup of $GL(d,R)$ where $R=\Q,\Z $ or $\F_q$.
This function tests if <G> is
solvable and returns 'true' or 'false'. 

\> IsPolycyclicMatGroup( <G> )

<G> is  a subgroup of $GL(d,R)$ where $R=\Q,\Z $ or $\F_q$.
 This function tries to test if <G> is polycyclic. If <G> is polycyclic,
 then it returns 'true'. 
If <G> is not polycyclic, then there are two cases: If $R=\Z$ or $\F_q$,
then the function returns 'false'. In case that $R=\Q$ the algorithm may
return 'false' or may not terminate.
 
%%%%%%%%%%%%%%%%%%%%%%%%%%%%%%%%%%%%%%%%%%%%%%%%%%%%%%%%%%%%%%%%%%%%%%%%%
\Section{Module series}

Let $G$ be a finitely generated solvable subgroup of $GL(d,\Q)$. The vectors
space $\Q^d$ is a modul for the algebra $\Q[G]$. The following
functions provide the possiblity to compute certain module series of
$\Q^d$. Recall that the radical $Rad_G(\Q^d)$ is definied to be the
intersection of maximal $\Q[G]$-submodules of $\Q^d$. Further the
radical series 
$$
0=R_n \< R_{n-1} \< \dots \< R_1 \< R_0=\Q^d 
$$
is definied by $R_{i+1}:= Rad_G(R_i)$. 

\> RadicalSeriesSolvableMatGroup( <G> )

return the radical series for the solvable rational matrix group
<G>. 

A modul is said to homogeneous if it is the direct sum of
irreducible and isomorphic submodules. 
A radical series of $\Q^d$ can be refinied to a homogeneous series. 
That is a submodule series such that the factors are homogeneous.

\> HomogeneousSeriesAbelianMatGroup( <G> )

returns the homogeneous series for the abelian rational matrix group <G>.

Further a homogeneous series can refined to a composition series. That is
a submodule series such that the factors are irreducible.

\> CompositionSeriesAbelianMatGroup( <G> )

returns the composition series for the abelian rational matrix group <G>.

%%%%%%%%%%%%%%%%%%%%%%%%%%%%%%%%%%%%%%%%%%%%%%%%%%%%%%%%%%%%%%%%%%%%%%%%%
\Section{Examples}

\> PolExamples( <l> )
 
Returns some examples for polycyclic rational matrix groups, where <l> 
is an integer
between 1 and 24. 
These can be used to test the functions in this package. 


%%%%%%%%%%%%%%%%%%%%%%%%%%%%%%%%%%%%%%%%%%%%%%%%%%%%%%%%%%%%%%%%%%%%%%%%%
%%
%E  Emacs . . . . . . . . . . . . . . . . . . . . . local emacs variables
%%
%%  Local Variables:
%%  fill-column:    73
%%  End:
%%












